\documentclass[12pt,letterpaper]{article}
\usepackage{fullpage}
\usepackage[top=2cm, bottom=4.5cm, left=2.5cm, right=2.5cm]{geometry}
\usepackage{amsmath,amsthm,amsfonts,amssymb,amscd}
\usepackage{lastpage}
\usepackage{enumerate}
\usepackage{fancyhdr}
\usepackage{mathrsfs}
\usepackage{xcolor}
\usepackage{graphicx}
\usepackage{listings}
\usepackage{hyperref}

\hypersetup{%
  colorlinks=true,
  linkcolor=blue,
  linkbordercolor={0 0 1}
}
 
\setlength{\parskip}{0.05in}

% Edit these as appropriate
\newcommand\course{CS 6620}

\pagestyle{fancyplain}
\headheight 35pt
\lhead{46 points}
\chead{\textbf{\Large Course Project}}
\rhead{\course \\ assigned: 03/24/2020\\ 
proposal due: 03/27/2020\\
written due: 05/01/2020
}
\lfoot{}
\cfoot{}
\rfoot{}
\headsep 1.5em

\begin{document}
~

The course project provides a context for you to study
an aspect of program analysis or compilation more deeply.
Projects must apply or extend concepts discussed in the class.
Projects must involve a substantial implementation and evaluation
component.  The project will be documented in a written
report whose details are described below. 

Projects can be done individually or in teams of two.

\section*{Part 1 (10 points)}
Prior to March 27, 2020 you should discuss your
project with the instructor
and formulate a description of the project
and a weekly work plan for how you will conduct the project.

A brief description of the project and your weekly plan should
be submitted, via email, to the instructor no later than
midnight on Friday March 27, 2020.

\section*{Part 2 (10 points)}
You will give a 25 minute presentation explaining your project findings
on April 21, 23, or 28.  

You will be graded on the organization of your presentation,
the quality of your presentation materials,
the clarity of your explanations,
the extent to which you use your 25 minutes appropriately (i.e., keeping
to the time limit, but also using the time appropriately),
and your ability to answer questions about your presentation.

There are many good resources available online about how to prepare
and give a technical presentation.  This site is a nice example:
\url{https://homes.cs.washington.edu/~mernst/advice/giving-talk.html}

\section*{Part 3 (26 points)}
Projects will vary greatly, so there is no one form for their
description.  Your report should take the form of a 
conference paper in ACM Standard format (\url{https://www.acm.org/binaries/content/assets/publications/consolidated-tex-template/acmart-master.zip}). The report must include an abstract, a motivation 
section that clearly defines the problem being solved,
a discussion of background and related work, a presentation
of the core contribution of the project, a section describing
the evaluation of the project outcomes, and a bibliography
with appropriate citations.

In addition, you are expected to submit any additional artifacts,
e.g., source code, test cases, etc., developed in your project
in a form that permits the instructor to ``replicate'' your work.
For example, if you extend the \texttt{tipc} compiler you must
provide everything necessary to build and test your extension.

You will be graded on the organization of the report,
the quality and clarity of the written content of the report,
the replicability of your project,
and quality of the technical work carried out on the project.

All project materials must be submitted no later than midnight on
May 01, 2020.

\end{document}
